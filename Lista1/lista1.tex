\documentclass[a4paper,12pt]{article}
\usepackage[brazilian]{babel}
\usepackage[utf8]{inputenc}

\renewcommand{\thesubsection}{\thesection.\alph{subsection}}

\title{CK0031 - Lista 1}
\author{Marcos Felipe De Menezes Mota - 354080}
\date{}

\begin{document}
\maketitle
\section{Questão 1}

\subsection{}
Falso, pois um agente com apenas informação parcial sobre o estado do mundo pode gerar um função de probabilidade para as ações. Com a nossa definição de racionalidade baseado em maximação de um função de desempenho, podemos dizer que um agente é perfeitamente racional mesmo se ele maximizar uma função probabilística de desempenho e ações.

\subsection{}
Verdadeiro, pois em ambientes sequenciais um agente reflex iria apenas agir baseado no percept atual mas a caracteristica desses ambientes são, que ações efeitas no presente pode afetar as ações a serem tomadas no futuro. Como agentes reflex não tem essa habildade, nunca poderam selecionar suas ações baseadas na melhora de uma madida de desempenho para esse ambiente.

\subsection{}
Verdadeiro. Um componente essencial de um task environment é a medida de performace. Logo se atribuirmos uma media de performace que sempre da a pontuação maxima para qualquer ação, qualquer agente vai ser racional.

\subsection{}
Falso. Um agent program tem como entrada um percept atual do ambiente. Já uma agent function tem como entrada o histórico de percepts.

\subsection{}
Verdadeiro. Pois podemos enumerar os históricos de percepts e as ações correspondentes em uma tabela, pode ser não eficiente mas sempre pode ser computada.

\subsection{}

Verdadeiro. Por exemplo usando uma medida de performace do item C obtemos esse tipo de resultado.

\subsection{}

Parece Falso.

\subsection{}

\subsection{}

Falso. Por mais que o agente tome sempre as decisões mais racionais possiveis o ambiente não pode ser completamente observável e fatores como blefe não permitem uma modelagem matemática determinística.

\section{Questão 2}
\subsection{Playing Soccer}
\subsubsection{PEAS}
\begin{table}[h!]
\begin{tabular}{l|l|l|l}
\hline
Performace & Environment & Actuators & Sensors \\
\hline
Número de Gols & Campo de Futebol & Pernas & Camera \\
Defesas & & Mãos & Sensor de distância \\
\end{tabular}
\end{table}
\subsubsection{Caracterização}
\begin{table}[h!]
\begin{tabular}{|c|c|c|c|c|c|c|}
\hline
partially observable & multi-agent & stochastic & sequential & dynamic & continuous & known
\end{tabular}
\end{table}

\subsection{Shop Books}
\subsubsection{PEAS}
\begin{table}[h!]
\begin{tabular}{l|l|l|l}
\hline
Performace & Environment & Actuators & Sensors \\
\hline
Descontos & Serviço de Compras Online & HTML parser & Listeners \\
Número de livros & & efetuar procedimentos & buscador\\
\end{tabular}
\end{table}
\subsubsection{Caracterização}
\begin{table}[h!]
\begin{tabular}{|c|c|c|c|c|c|c|}
\hline
fully observable & single agent & deterministic & episodic & dynamic & discrete & known
\end{tabular}
\end{table}

\subsection{Tennis Match}
\subsubsection{PEAS}
\begin{table}[h!]
\begin{tabular}{l|l|l|l}
\hline
Performace & Environment & Actuators & Sensors \\
\hline
Sets Ganhos & Quadra de Tennis & Mãos & Camera \\
Número de pontos & & Motores & Sensor de distância\\
Velocida de lançamento & & Raquete & osciloscópio\\
\end{tabular}
\end{table}
\subsubsection{Caracterização}
\begin{table}[h!]
\begin{tabular}{|c|c|c|c|c|c|c|}
\hline
partially observable & multi-agent & stochastic & sequential & dynamic & continuous & known
\end{tabular}
\end{table}

\subsection{Tennis Match Against Wall}
\subsubsection{PEAS}
\begin{table}[h!]
\begin{tabular}{l|l|l|l}
\hline
Performace & Environment & Actuators & Sensors \\
\hline
Sets Ganhos & Quadra de Tennis & Mãos & Camera \\
Número de pontos & & Motores & Sensor de distância\\
Velocida de lançamento & & Raquete & osciloscópio\\
\end{tabular}
\end{table}
\subsubsection{Caracterização}
\begin{table}[h!]
\begin{tabular}{|c|c|c|c|c|c|c|}
\hline
fully observable & single-agent & stochastic & sequential & dynamic & continuous & known
\end{tabular}
\end{table}

\subsection{High Jump}
\subsubsection{PEAS}
\begin{table}[h!]
\begin{tabular}{l|l|l|l}
\hline
Performace & Environment & Actuators & Sensors \\
\hline
Altura & Mundo & Pernas & Camera \\
 & & Motores & Sensor de distância\\
 & & Armortecedores & osciloscópio\\
\end{tabular}
\end{table}
\subsubsection{Caracterização}
\begin{table}[h!]
\begin{tabular}{|c|c|c|c|c|c|c|}
\hline
fully observable & single-agent & stochastic & episodic & dynamic & continuous & known
\end{tabular}
\end{table}

\subsection{Biddin at an Auction}
\subsubsection{PEAS}
\begin{table}[h!]
\begin{tabular}{l|l|l|l}
\hline
Performace & Environment & Actuators & Sensors \\
\hline
Altura & Mundo & Pernas & Camera \\
 & & Motores & Sensor de distância\\
 & & Armortecedores & osciloscópio\\
\end{tabular}
\end{table}
\subsubsection{Caracterização}
\begin{table}[h!]
\begin{tabular}{|c|c|c|c|c|c|c|}
\hline
fully observable & single-agent & stochastic & episodic & dynamic & continuous & known
\end{tabular}
\end{table}


\end{document}
